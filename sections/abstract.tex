\begin{abstract}
This paper presents EinsteinPy (version 0.3), a community-developed Python
package for gravitational and relativistic astrophysics. Python is a free, easy to use high level programming language which has seen a huge expansion in the number of its users and developers in recent years. Specifically, a lot of recent studies show that the use of Python in Astrophysics and in general physics has increased exponentially. Many great frameworks came as Python packages which provide a very high level of abstraction over the dirty nitty-gritty of complex algorithms and provide an easy to use interfaceand pleasing user experience. One such example is Keras - framework for deep learning which has made deep learning so easy that a person with zero programming knowledge can also train a neural network classifier. This example really demonstrates the power of abstraction which is achievable in Python. The aim of the EinsteinPy is no different and is developed keeping in mind the state of a theoretical gravitational physicist with a little or no background in computer programming and trying to work in the field of numerical relativity or trying to use simulations in their research. Currently EinsteinPy supports simulation of time-like and null geodesics and calculate trajectories in different background geometries some of which are Schwarzschild, Kerr and KerrNewmann along with coordinate inter-conversion pipeline. It has a partially developed pipeline for plotting and visualization with dependencies on libraries like plotly, matplotlib etc. One of the unique feature of EinsteinPy is a sufficiently developed symbolic tensor manipulation utilites which is a great tool in itself for teaching yourself tensor algebra which for many beginner students can be overwhelmingly tricky. Currently EinsteinPy also provide few utility functions for hypersurface embedding of Schwarzschild spacetime which further will be extended to model gravitational lensing simulation. The current version of the library is in a state that can be used by any serious student of general relativity trying to get essence of this beautiful subject but is somewhere lost in the heavy mathematical formalism of the subject. EinsteinPy provides such students to really see through the equations and visualize whats really happening.
\end{abstract}
